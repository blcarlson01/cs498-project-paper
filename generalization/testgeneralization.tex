
%
%  $Description: Author guidelines and sample document in LaTeX 2.09$
%
%  $Author: ienne $
%  $Date: 1995/09/15 15:20:59 $
%  $Revision: 1.4 $
%

\documentclass[times, 10pt,twocolumn]{article}
\usepackage{latex8}
\usepackage{times}

%\documentstyle[times,art10,twocolumn,latex8]{article}

%-------------------------------------------------------------------------
% take the % away on next line to produce the final camera-ready version
\pagestyle{empty}

%-------------------------------------------------------------------------
\begin{document}

\title{An Empirical Study of Test Generalization in an Open Source Project}

\author{Xiaokang Xiang\\
Department of Computer Science\\
University of Illinois at Urbana-Champaign\\
Urbana, IL 61801\\xxiang4@illinois.edu\\
\and
Brandon Carlson\\
Department of Computer Science\\
University of Illinois at Urbana-Champaign\\
Urbana, IL 61801\\blcrlsn2@illinois.edu\\
}

\maketitle
\thispagestyle{empty}

\begin{abstract}
Note that this sample paper skeleton serves as a guideline for
helping produce a high-quality project report. If you think that you
have a better or alternative way for presenting your project results
(not fully following the guideline here), you shall contact the
instructor.

In the abstract, you shall describe briefly the motivation of test
generalization, the name and characteristics of your open source
project under test, and the findings of your empirical study.

\end{abstract}

%-------------------------------------------------------------------------
\Section{Introduction} \label{sec:intro}

You shall describe with more details on the background of
parameterized unit tests
(PUTs)~\cite{tillmann05:parameterized,tillmann06:unit} and
Pex~\cite{tillmann08:pex}, the motivation of test generalization
(e.g., despite the benefits of PUTs, many existing unit tests are
not written in PUTs), how you can generalize conventional unit tests
(recall what the instructor taught in class), the name and
characteristics of your open source project under test, and the
findings of your empirical study.

You shall list the structure of the rest of the paper.

%-------------------------------------------------------------------------
\Section{Example} \label{sec:example}

You shall give an example conventional unit test (which is preferred
to be from your open source project under test) and illustrate the
procedure of test generalization with this example.

%-------------------------------------------------------------------------
\Section{Open Source Project Under Test} \label{sec:subject}

You shall give details on what the project is about, and the
characteristics of the project's production code base (such as the
number of classes, the number of methods, the number of public
methods, and the lines of code), and the project's test code base
(such as the number of test classes, the number of test methods, and
the lines of test code). Note that you may use some existing tools
for producing these metrics; use your Google skills to find out
these tools for you to use!

If you do not generalize all the conventional test cases for the
open source project, you should list the characteristics of the
portion that you focus on (e.g., showing the similar metrics listed
above for the focused portion). You may give some justification on
why you focus on your selected portion instead of other conventional
test classes.

%-------------------------------------------------------------------------
\Section{Benefits of Test Generalization} \label{sec:benefits}

You shall describe the percentage of conventional unit tests that
you can generalize to PUTs and the percentage of convectional unit
tests that you cannot.

For those that are amenable to test generalization, you shall list
and compare the code coverage achieved by the original conventional
unit tests and your generalized PUTs (together with their generated
tests by Pex). If you observe some new abnormal behaviors indicating
potential faults (such as new uncaught exceptions or unexpected
assertion violations), you can also describe them as benefits of the
PUTs. When you list code coverage, you can list details (test class
by test class or test method by test method) or just list the
aggregated statistics for all the test classes or test methods. How
much detailed you want get into depends on the page limit and how
you want to devote the limited space for showing the best (most
interesting) results or findings from your course project.

%-------------------------------------------------------------------------
\Section{Categorization of Conventional Unit Tests}

\SubSection{Conventional Unit Tests Amenable to Test Generalization}
\label{sec:amenable}

You shall categorize your PUTs generalized from conventional unit
tests into the test patterns proposed by de Halleux and
Tillmann~\cite{halleux08:putpatterns}. You shall list the statistics
of your PUTs falling into each pattern.

You shall also propose new patterns to accommodate those PUTs that
you cannot categorize into any of the patterns proposed by de
Halleux and Tillmann~\cite{halleux08:putpatterns}. You shall
describe the definition of these new test patterns and the example
PUTs for these patterns. If you run out of space, you can refer the
readers to the wiki entry links for the details of your new test
patterns. Note that you shall at the same time prepare your wiki
entries for your new test patterns no matter whether you describe
your new patterns here in details.

You shall list the statistics of your PUTs falling into each of your
new pattern being proposed by you.

If you cannot find any conventional unit test to be amenable to test
generalization, you can state so (but you are expected to find at
least one conventional unit test to be amenable to test
generalization).


%-------------------------------------------------------------------------
\Section{Conventional Unit Tests Not Amenable to Test
Generalization} \label{sec:notamenable}

You shall summarize the types of conventional unit tests that are
not amenable to test generalization. It would be better if you can
categorize them into anti-generalization patterns (if so, you shall
also create wiki entries for your anti-generalization patterns).



\Section{Helper Techniques for Test Generalization}
%-------------------------------------------------------------------------
\SubSection{Factory Methods} \label{sec:factory}

You shall summarize the cases where you use factory methods to help
Pex to generate better test inputs for your generalized PUTs. Again,
it would be better if you can summarize patterns or categories for
these cases.

If you find no such cases, you can state so.

%-------------------------------------------------------------------------
\SubSection{Mock Objects} \label{sec:mock}

You shall summarize the cases where you use mock objects to deal
with some complications that you face in test generalization. You
shall categorize these cases into the mock object patterns proposed
by de Halleux and Tillmann~\cite{halleux08:putpatterns}. If you
cannot categorize them into these patterns, propose new patterns and
document them in wiki similar to the preceding guidelines for
documenting your new normal PUT patterns.

If you find no such cases, you can state so.


%-------------------------------------------------------------------------
\SubSection{New Assertions} \label{sec:assertions}

You shall summarize the cases where you add new assertions to the
generalized PUTs in order to improve the PUTs to capture more
behaviors or properties.

If you find no such cases, you can state so.

%-------------------------------------------------------------------------
\Section{Limitations of Pex or PUTs} \label{sec:limitations}

You shall summarize the limitations of Pex in terms of supporting
your test generalization or test generation for your PUTs. You shall
also summarize the limitations of PUTs (e.g., some behavior cannot
easily be expressed in PUTs and call for new types of tests or
specification forms), which could be related to cases for not being
amenable to test generalization described earlier.

If you find no such cases, you can state so.

%-------------------------------------------------------------------------
\Section{Conclusion}

Your conclusion can be structured similar to the structure of your
abstract. But do not just copy your abstract here.

%-------------------------------------------------------------------------

\bibliographystyle{latex8}
\bibliography{pex}

\end{document}
